\documentclass[12pt]{article}


\usepackage{fullpage} %full page typesetting
\usepackage{setspace} %allows for non-singlespacing
\usepackage{graphicx} %graphics capabilities
\usepackage{latexsym} %extra symbols
\usepackage{rotating} %rotation for figures
\usepackage{longtable} %tables that fill more than a single page
\usepackage{hyperref} %hypertext links in the document
\usepackage{natbib} %better bibliographies
\usepackage{authblk} %author and affiliation in opening
\usepackage{mathpazo} %use palatino font, rather than times

\doublespacing

\title{\tb{Place of Residence and Political Attitudes in Democracies Worldwide \\ {\large Research Prospectus} }}

\author{Jennifer Lin}
\affil{Transitions to Democracy}

\newcommand\e{\emph}
\newcommand\tb{\textbf}
\newcommand\un{\underline}

\begin{document}

\begin{singlespace}
\maketitle
\end{singlespace}

\section{Guiding Question}

This research project is interested in understanding the influence of an individual's place of residence (rural, suburban or urban) on their political attitudes. The research aims to understand the ways in which context shapes political ideology. By comparing countries with differing electoral contexts, varying stages of transition to democracy, and differing consolidations of democratic institutions, I hope to understand the influence that an individual's place of residence and their broader political environment shape their electoral choices.

Put simply, the research aims to answer the following questions:
\begin{enumerate}
\item Does place of residence (rural, suburban, or urban) influence individual's political attitudes? If so, how?
\item Does difference in wealth between place of residence work to influence these differences between political ideology?
\item How does the country's electoral context influence these patterns? Does the variation between plurality or proportional electoral systems influence this relationship?
\item How does the country's age of democracy (if successfully transitioned in the past), stage of transition to democracy (if currently transitioning), or experiences of failure with transition influence these patterns? In other words, if a relationship between place of residence and vote choice exists, is it more pronounced when the individual lives in an established democracy or a transitioning democracy? 
\item If data is available, do these patterns persist over time? Are they enduring or do they change after a certain period of time? If they change, what caused them to change?
\end{enumerate}

\section{Preliminary Bibliography}

\renewcommand{\bibsection}{}
\nocite{*}
\bibliography{RuralPolitics}
\bibliographystyle{chicago}

\section{Data}

The data of this study will originate from the CSES Modules I-IV (available at \url{http://www.cses.org/datacenter/download.htm}) survey data and will rely on variables related to respondent place of residence and its effect on their political outlook.

\begin{itemize}
\item Place of residence is based on individual self-identified place of residence. I may also consider self-identified socioeconomic status to understand any differences between rural and urban wealth on personal political outlook.
\item Political outlook will be measured by the individual's self placement on the left-right scale along with their vote choices for the most recent presidential election.
\item The regime will be classified by the type of electoral system, electoral formula, age of regime (number of years since the most recent regime change), and level of democracy (whether it is closer to a democracy or autocracy). I will also aim to create a variable that identifies the regime's stage in transition to democracy and classify it based on current political developments and historic trends.
\item Changes over time will depend on availability in CSES data. If a country is present over multiple modules, I will analyze the variables based on similar variables to establish a pattern (if one exists) to understand how these variables interact and change over time.
\end{itemize}

I recognize some limitations regarding availability with CSES data given that the earliest data entry is in 1996, which is late in the third wave of democracy. Therefore, for this project, I am more interested in the influence of a country's transition to democracy on political attitudes of its residents. Through a careful reading of the literature, I hope to find other sources of data that can tap into how the process of democratic transition influences individual political ideology. However, as of right now, I will work with what the CSES can offer and build upon it with more information that I may find. 

\section{Expected Findings}

In the 2016 election, rural America was crucial in helping Donald Trump carry a victory to the Presidency. We can infer this based on research done in rural Wisconsin for an earlier gubernatorial race \citep{walsh_putting_2012}. The rural consciousness leads people to overwhelmingly vote Republican, or for the conservative (right-wing) party in an election, and should boost this rate of support when the candidate pushes for reforms that would appeal to their "work hard for what you want" mindset. Therefore, I anticipate that other places in the world would be no different. Citizens in established democracies, no matter the electoral pattern, would see rural residents be more conservative in their attitudes towards politicians and policies. However, countries that are still transitioning to democracy may produce mixed results, which may depend on the wealth divide between rural and urban residents. If the electoral system persists over time, this pattern may continue to exist in the regime but may be influenced by other extraneous variables not considered by this study. Overall, I expect to see a pattern between place of residence, SES (if applicable), and conservative voting patterns. The electoral system of the country and age/transition stage of regime may also influence these patterns but the association of these variables may also vary based on the context of the election and governing system of the regime.

\end{document}