\documentclass[12pt, titlepage]{article}

\usepackage{fullpage} %full page typesetting
\usepackage{setspace} %allows for non-singlespacing
\usepackage{graphicx} %graphics capabilities
\usepackage{latexsym} %extra symbols
\usepackage{rotating} %rotation for figures
\usepackage{longtable} %tables that fill more than a single page
\usepackage{hyperref} %hypertext links in the document
\usepackage{natbib} %better bibliographies
\usepackage{authblk} %author and affiliation in opening
\usepackage{mathpazo} %use palatino font, rather than times

\title{\tb{Place of Residence and Political Attitudes in Democracies Worldwide }}

\author{Jennifer Lin}
\affil{New College of Florida}

\newcommand\e{\emph}
\newcommand\tb{\textbf}
\newcommand\un{\underline}
\newcommand\txt{\texttt}

\doublespacing 

\begin{document}

\begin{singlespace}
\maketitle
\end{singlespace}

\begin{center} %center text
\section*{Abstract} %the star stops this section from appearing in the ToC
	
\begin{quote}
After the 2016 election, many major new pundits pointed at Rural America as the major source of Donald Trump’s victory for the presidency. This is because, as prior research by \cite{walsh_putting_2012} for the United States and \cite{walks_city-suburban_2005} for Canada suggests, residents in rural parts of democratic countries tend to be more religious and less exposed to diversity as their urban counterparts. As a result, they are more conservative in their social views even if they may be more liberal on their fiscal ones. President Trump took advantage of this and won the hearts and minds of many voters in the Midwest and Rural South, which allowed him to secure an electoral college victory despite falling short on the popular vote. For this research project, I was interested to see if this trend extends to other developing and consolidated democracies in the world. Using the CSES Module IV data (2011-2016), I conducted regressions to test place of residence on a self-defined ideological scale and an objective ideology scale constructed using respondent opinions on various issues. By controlling for regime factors such as age and level of democracy, I find statistical significance to suggest that place matters. The data provides sufficient evidence to suggest that if one lives in an urban place compared to a rural one, they are more likely to self-identify as more liberal and hold those values when voicing their opinions on social issues.

\end{quote}
\end{center}

\clearpage

\section{Introduction}



\section{Theoretical Review of Literature}


\section{Hypotheses}

From the review of literature, we see that there seems to be an ideological divide between people living in rural and urban areas of their respective countries. From case studies, most notably in the United States and Canada, there is a pattern such that the individuals living in rural areas tend to err towards conservatism and residents of urban centers tend to veer towards liberalism. However, given that these are relatively established democracies, I wonder how differences in levels of democratic development, regime age and other ideosyncratic factors that shape a regime influence the divide of political ideologies between urban and rural residents. Therefore, the present research will test three hypotheses relating to the relationship between the place of residence of an individual and their political ideology.

\begin{quote}
	\e {Hypothesis 1 -- Place Matters:} An individual's place of residence influences their political ideology such that the more distant the characteristics of one's vicinity gets from an urban city, the more conservative one will become.
\end{quote}

The \e{Place Matters} hypothesis suggests that there is a direct relationship between place of residence and political ideology no matter where an individual lives in this world. Therefore, this hypothesis assumes that, broadly speaking, individuals who live in a rural area will be more conservative than those who live in a small town or urban center. 

\section{Research Design}

\subsection{Data}

To understand the divide between rural and urban residents in different countries of the world, I will consider cases analyzed in the Comparative Study of Electoral Systems (CSES) Module IV dataset for the analysis of these trends. In this specific module, the CSES considers elections that occurred between 2011 and 2016 in various democracies around the world. 

\subsection{Key Independent Variables}

In this data analysis, the key independent variables that are considered include the respondent's country and place of residence. For this analysis, place of residence is considered to be based on the categories that are established by the survey such that individuals either live in a rural village, small town, suburbs of a large city or within the large city itself. In the regression model, this variable is characterized as a category even though there are places that may be in between two of the categories that characterize place of residence. 

Each individual's country of residence is also classified based on the year of the election, level of democracy in the country,  age of the current governing regime in the country at the time of the election, and the type of electoral formula employed by the governing body to determine winners of seats in government. The level of democracy is characterized by how free the polity is at the time of the election. This measure was gathered from the generators of the Polity IV project and reflects freedom from a -10 to 10 scale with -10 being the most autocratic and 10 being the most democratic. The age of the current governing regime suggests how long the current system has been in power. The larger value suggests that the current system in government has been established for a longer period of time and is therefore more stable. Finally, the type of electoral formula describes how government official win office based on their total vote share. Contestants may win via majoritarian, proportional or mixed system. 

\subsection{Key Dependent Variables and Measures}

To understand where individuals see their own political values, the dependent variable reflects their self-identified ideology, a variable coded by the CSES that ranges from 0 to 10 with 0 being the most left leaning and 10 as the most right leaning ideological stance.. Since this can be rather subjective, i also created a liberalism scale that reflects individual values on government spending with a more liberal vision being those who value spending for social benefits and a more conservative view for those who do not with for such spending or for those who wish to spend more to boost defense as a means to secure their country's identity and values. This 9-point scale was generated from questions relating to public expenditure used in the survey. Topics for these questions include health, education, unemployment benefits, defense, old age pensions, business and industry investments, police and law enforcement and welfare benefits. Additionally, I add a value of economic equality to the measure as a means of understanding how each person values equality as part of their ideological outlook. This scale ranges from 0 to 9, with 0 being the most conservative and 9 being the most liberal. In the broadest context, respondents who score at the extreme ends of the scale are seen to be most conservative or liberal in terms of their outlook on spending, which can speak to their views on social values given their willingness to allocate government money to these groups.

\section{Results}


\section{General Discussion and Conclusion}


\clearpage
\bibliography{RuralPolitics}
\bibliographystyle{chicago}


\end{document}
