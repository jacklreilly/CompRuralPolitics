\documentclass[12pt, titlepage]{article}

\usepackage{fullpage} %full page typesetting
\usepackage{setspace} %allows for non-singlespacing
\usepackage{graphicx} %graphics capabilities
\usepackage{latexsym} %extra symbols
\usepackage{rotating} %rotation for figures
\usepackage{longtable} %tables that fill more than a single page
%\usepackage{hyperref} %hypertext links in the document
\usepackage{natbib} %better bibliographies
\usepackage{authblk} %author and affiliation in opening
\usepackage{mathpazo} %use palatino font, rather than times
\usepackage{appendix}
\usepackage{lscape}
\usepackage{tabulary}
\usepackage[nottoc]{tocbibind}
\usepackage[colorlinks=true,linkcolor=blue,citecolor=cyan]{hyperref}
\usepackage{ifthen}
\usepackage{float}


\title{\tb{Place of Residence and Political Attitudes in Democracies Worldwide \\ {\large Online Appendix A -- Original Results} }}

\author{Jennifer Lin}
\affil{New College of Florida}

\newcommand\e{\emph}
\newcommand\tb{\textbf}
\newcommand\un{\underline}
\newcommand\txt{\texttt}

\doublespacing 

\begin{document}

\begin{singlespace}
\maketitle
\end{singlespace}

\section{Results for Self Placement Ideology - Original}

\subsection{Self-Placement Ideology as a Dependent Measure}

\subsubsection{\e{Place Matters} on the Worldwide Scale}

For the first model, a regression was conducted to see if there is a general trend between place and self-placement on a 10-point ideology scale. Furthermore, the regression was conducted separately for each election that was represented in this dataset. Table \ref{table3} shows the results for this regression on the global scale. This general trend suggests that, without considerations of any polity-specific factors, there is a difference between the global rural residents, small town and suburban residents in terms of their political ideologies. The trends described in this table suggests that, on a general level, there lies a difference between where people live and their political attitudes. However, several caveats are present in this analysis. From here, we are assuming that individuals conceptualize the places of residence in a similar fashion across geographic boundaries. Additionally, this analysis does not take into account country-specific measures such as the presence of the different possibilities of residence for the people. As I noted in Table \ref{table2}, not all the countries have respondents from each place of residence as categorized in the CSES. Therefore, it would help to break down the analysis by country.

\begin{singlespace}
	\begin{table}[H]
		\centering
		%\def\arraystretch{1.5}
		\caption{\tb{Self-Placement Ideology - Worldwide}}
		\begin{tabulary}{\linewidth}{l c}
			\\
			\hline
			\tb{Place of Residence}&\tb{Worldwide} \\
			\hline
			Small Town&-.1338***  \\    
			& (.0336)   \\
			Suburban & -.2591***\\ 
			& (.0358) \\
			Urban   & -.0482   \\
			& (.0300)    \\
			Constant   & 5.597***  \\
			&(.0234) \\
			N  & 47,821  \\
			$R^2$	& 0.0011 \\
			\hline                                       
		\end{tabulary}
		\\
		\e{Notes:} *p$<$.1, **p$<$.05. ***p$<$.01 \\
		\e{Reference:} A rural place of residence serves as the baseline for comparison
		\label{table3}
	\end{table}
\end{singlespace}

\subsubsection{\e{Place Matters} By Geographic Region}

When the analysis is broken down by the countries themselves, we see a clearer patterns between place of residence and political attitudes of the respondents to the CSES survey. Each country is displayed and analyzed with their regional neighbors. Additionally, these trends are discussed in context of some key macro variables present for each polity that is displayed in Figure \ref{figure2} and Appendix \ref{AppendixC}.

Table \ref{table4} discusses the patterns seen by place of residence for countries in Central America and Latin America. From the results of the regression, we see that urban residents in Argentina are, on average, more liberal than their rural counterparts. This is slightly the case for Mexico at the time of their 2015 elections, but we do not see these trends elsewhere. When considering the levels of democracy, electoral formula and regime age for these countries, the notable difference between the regimes is the age of the system. Residents of Argentina have been living in a democracy for a longer period of time than the other countries represented in this region. 

\begin{singlespace}
	\begin{table}[H]
		\centering
		%\def\arraystretch{1.5}
		\caption{\tb{Self-Placement Ideology - Central/Latin America}}
		\begin{tabulary}{\linewidth}{l c c c c}
			\\
			\hline
			\tb{Place of Residence}&\tb{Argentina}&\tb{Mexico (2012)}&\tb{Mexico (2015)} &\tb{Peru}\\
			\hline
			Small Town  & -    & .6408***   &-.4734    & -   \\      
			& -  & (.2394) & (.3113)    & -    \\
			Suburban    & -    & -   & -   & -    \\ 
			& -     & -    & -   & -    \\
			Urban   & -.9521*** & -.1417   & -.4084*   & .0883  \\
			& (.2150)   &(.1678)   & (.2210)  & (.1743)      \\
			Constant   & 6.495*** & 6.6516*** & 6.1690*** & 6.646***   \\
			&(.2046)&(.1493)&(.1998)&(.1564) \\
			N  & 1,241 &1,812   & 910 & 1,438   \\
			$R^2$ & 0.0156   &0.0082   &0.0041 &0.0002     \\
			\hline                   
		\end{tabulary} 
		\\
		\e{Notes:} *p$<$.1, **p$<$.05. ***p$<$.01 \\
		\e{Reference:} A rural place of residence serves as the baseline for comparison
		\label{table4}
	\end{table}
\end{singlespace}

In Western Europe, as Table \ref{table5} shows, there is a clearer divide between place of residence and political ideologies, such that, for all of the counties displayed in the table, urban residents are significantly more likely to be more left-leaning than their rural counterparts. Additionally, suburban residents in Switzerland and Portugal are also more likely to be more left-leaning than their rural counterparts. Small town residents can be slightly more left-leaning than rural residents in Portugal, Great Britain and Germany. The countries in this region are some of the free-est and oldest democracies in the world. Therefore, individuals here have had a longer time to socialize and develop their opinions based on factors relating to self-interest. 

\begin{landscape}
	\begin{table}
		\centering
		\def\arraystretch{1.5}
		\caption{\tb{Self-Placement Ideology - Western Europe}}
		\begin{tabulary}{\linewidth}{l c c c c c c}
			\\
			\hline
			\tb{Place of Residence}&\tb{Switzerland}&\tb{Germany}&\tb{France} &\tb{Great Britain}&\tb{Ireland}&\tb{Portugal}\\
			\hline
			Small Town   & - .4973  & -.3295*** &-.1148 &  -.3773***  & -   &-.3717* \\      
			& (.4333)  & (.1087)  & (.1321)    & (.1392)  &- & (.2157) \\
			Suburban  & -.2663***   & .0912 & -.0246  & -      &-    & -1.1139*** \\ 
			& (.0852)  & (.1693)  & (.1886)   & -      & - & (.2421)  \\
			Urban  & -.8650***  & -.6361*** & -..7254***  & -.8310***      &-.2494**& -.8348***   \\
			& (.0976)  &(.1311)  & (.1706)    & (.1538)   & (.0986)  & (.2121)  \\
			Constant & 5.462***  & 4.630***  & 4.8086***  & 5.3992***  & 6.1491***  &5.4661***   \\
			&(.0681)&(0.0845)&(.0943)&(.1197)&(.0772)&(.1483) \\
			N  & 4,728    &1,697  & 1,916  & 1,386     &  1,594  &1,224  \\
			$R^2$  & 0.0191    &0.0176 &0.0102    &0.0217  &  0.0040 &0.0218  \\
			\hline                                       
		\end{tabulary} 
		\\
		\e{Notes:} *p$<$.1, **p$<$.05. ***p$<$.01 \\
		\e{Reference:} A rural place of residence serves as the baseline for comparison
		\label{table5}
	\end{table}
\end{landscape}

Countries in Northern Europe are also relatively free in terms of their level of democracy and time that such democracy is established. Like their Western European counterparts, Scandinavian countries hold similar governing systems but do not experience similar trends in terms of place of residence and political attitudes. As Table \ref{table6} suggests, Finland is the only country with a sharp urban-rural divide. Iceland's place divide comes with the suburban-rural split between the residents. While there are subtle levels of significance elsewhere, it is clear that there is a regional difference when it comes to how democracy influences individual's political attitudes based on place. In the light of countries of Western Europe, Northern European countries do not see the same patterns fo residence divides even with similar freedoms and levels of democracy. This suggests that there are other differences in these regions that are worth considering.

\begin{singlespace}
	\begin{table}[H]
		\centering
		%\def\arraystretch{1.5}
		\caption{\tb{Self-Placement Ideology - Scandinavia}}
		\begin{tabulary}{\linewidth}{l c c c c}
			\\
			\hline
			\tb{Place of Residence}&\tb{Finland}&\tb{Iceland}&\tb{Norway}&\tb{Sweden} \\
			\hline
			Small Town&.1418&.5194**&.0568&-.1030 \\
			&(.1646)&(.2279)&(.1534)&(.2644) \\
			Suburban&-.0612&.7355***&-.0277&.0794 \\
			&(.1452)&(.2290)&(.1847)&(.2690) \\
			Urban&-.7128***&.3840*& .0888&-.1723\\
			&(.2202)&(.2220)&(.1429)&(.2368) \\
			Constant&5.6503***&4.990***&5.5985***&5.2230*** \\
			&(.1169)&(.2006)&(.1080)&(.1973) \\
			N&1,387&1,266&1,618&792\\
			$R^2$&0.0111&0.0096&0.0004&0.0018 \\
			\hline
		\end{tabulary}
		\\
		\e{Notes:} *p$<$.1, **p$<$.05. ***p$<$.01 \\
		\e{Reference:} A rural place of residence serves as the baseline for comparison
		\label{table6}
	\end{table}
\end{singlespace}

Turning to Central Europe, Table \ref{table7} suggests that there is a significant difference between place of residence and political attitudes in Slovenia and Slovakia. When we consider factors such as level of democracy and regime age as we did previously, freedom by democracy and stability of the regime are not necessarily trends that help shape the influence of place of residence and political ideologies here. As we previously established, it seems that the oldest regimes in each region lead to more pronounced differences between rural and urban residents, but this trend is not necessarily the case here. Austria is the oldest regime in the region, but urban residents are only slightly more left-leaning than their rural counterparts. 

\begin{singlespace}
	\begin{table}[H]
		\centering
		%\def\arraystretch{1.5}
		\caption{\tb{Self-Placement Ideology - Central Europe}}
		\begin{tabulary}{\linewidth}{l c c c c c}
			\\
			\hline
			\tb{Place of Residence} &\tb{Austria}&\tb{Czech Republic}& \tb{Poland} &\tb{Slovakia}&\tb{Slovenia} \\
			\hline
			Small Town&-.2108& -.0223 & .0829 & .4318*** & .0229 \\
			&(.1540) & (.1687) & (.1421) & (.2000) & (.3262)\\
			Suburban&-.1433& -.6091*& -.1414 & 1.059& .1549** \\
			&(.1570)&(.3281) & (.2210) & (.6711) & (.2491)\\
			Urban&-.3005**&.2860 & -.2002 & .8061*** & -1.3416***\\
			&(.1335) &(.1773) & (.2042) & (.2437) & (.3169)\\
			Constant& 5.3726***& 4.9135***& 6.015*** & 4.7527***  & 4.6780 ***\\
			&(.1219) & (.1300) & (.1050) & (.1370) & (.2060)\\
			N&3,363& 1,412& 1,634 & 879& 666 \\
			$R^2$&0.0018&0.0065 & 0.0016 & 0.0148 & 0.0629\\
			\hline
		\end{tabulary}
		\\
		\e{Notes:} *p$<$.1, **p$<$.05. ***p$<$.01 \\
		\e{Reference:} A rural place of residence serves as the baseline for comparison
		\label{table7}
	\end{table}
\end{singlespace}

In Southwestern Europe, the countries, especially those represented by two separate elections, suggest that trends in the influence of place of residence vary by election. This pattern is perhaps the clearest to discern in this region given that there are more countries here with multiple elections represented in the present module of the dataset. In Greece and Romania, we see that the urban-rural divide was present in the former but not the latter of the elections. This can be attributed to other factors such as candidates who are running in that particular race or the policies that are the most salient in the region at the time. Table \ref{table8} shows these trends and also suggests that Bulgaria and Montenegro see trends of rural-urban division.

\begin{landscape}
	\begin{table}
		\centering
		\def\arraystretch{1.5}
		\caption{\tb{Self-Placement Ideology - Southwestern Europe}}
		\begin{tabulary}{\linewidth}{l c c c c c c c}
			\\
			\hline
			\tb{Residence}& \tb{Bulgaria}& \tb{Greece ('12)}& \tb{Greece ('15)} & \tb{Montenegro} & \tb{Serbia} & \tb{Romania ('12)} & \tb {Romania ('14)}\\
			\hline
			Small Town&.5324 &-.7220*** &-.1458 & -1.3499*** & .2226 & .1211 & .2699\\
			&(.3313)& (.2656) & (.3484) & (.3282) & (.2098)& (.2304) & (.3097)\\
			Suburban& .7821*** & -1.058*** &-.0685 & -.9444** &.1046 &-.1422 &.1463\\
			& (.2914) &(.2557) & (.3667) & (.4876) &(.2900) & (.5737) & (.4000)\\
			Urban& .8838**& -1.1088*** & -.1737 & -1.4676*** & -.1549 & .7782*** & -.3691\\
			& (.3590)& (.1797) & (.3236) & (.4987) & (.2437) & (.2105) & (.3712)\\
			Constant& 4.739*** &5.5410*** & 4.5961*** &6.7329*** & 5.4643*** & 4.7229*** & 6.4406***\\
			& (.2267) & (.1271)  & (.3089) & (.2409) & (.1239) & (.1438) & (.1890)\\
			N& 750 & 928 & 905 & 450 & 1,094 & 1,192 &714\\
			$R^2$&0.0118 &0.0441& 0.0005 &0.0419 & 0.0020 & 0.130 &0.0037 \\
			\hline
		\end{tabulary}
		\\
		\e{Notes:} *p$<$.1, **p$<$.05. ***p$<$.01 \\
		\e{Reference:} A rural place of residence serves as the baseline for comparison
		\label{table8}
	\end{table}
\end{landscape}

Yet, not all countries experience patterns of changing urban-rural division between elections. In Eastern Europe, as represented by Latvia in Table \ref{table9}, the country did not see a shift in attitudes for respondents based on a change in election. 

\begin{singlespace}
	\begin{table}[H]
		\centering
		%\def\arraystretch{1.5}
		\caption{\tb{Self-Placement Ideology - Eastern Europe}}
		\begin{tabulary}{\linewidth}{l c c}
			\\
			\hline
			\tb{Place of Residence} & \tb{Latvia (2011)} &\tb{Latvia (2014)} \\
			\hline
			Small Town &.1635 &.0560 \\
			&(.2433) & (.2224) \\
			Suburban&- & -.1285 \\
			& - & (.2893) \\
			Urban&-.2292&.0221 \\
			& (.1850) & (.1996) \\
			Constant & 6.2775*** &6.3285*** \\
			& (.1360) & (.1396) \\
			N&787 & 815\\
			$R^2$& 0.0041 &0.0005 \\
			\hline
		\end{tabulary}
		\\
		\e{Notes:} *p$<$.1, **p$<$.05. ***p$<$.01 \\
		\e{Reference:} A rural place of residence serves as the baseline for comparison
		\label{table9}
	\end{table}
\end{singlespace}

For the data that is available to the Middle East, Africa and Asia, there are simply not enough cases here to make general arguments about the region as a whole. Nonetheless, they are still useful in helping us to understand which countries have a sharper urban-rural divide.

To start, Table \ref{table10} highlights Israel as a country in the Middle East with sharp urban and suburban divides to the rural areas of the country in terms of its political ideology. This is not an effect that is seen in Turkey. A reason for this trend is the freedoms that the people in these countries experience. While it is hard to tell if this pattern is true for all countries in the region, it is worth noting that the differences in the level of democracy here may have a great influence in individual political attitudes. Like the actors in Western Europe and much of the western, democratized world, Israel stands as one of the freest nations in the region. Their democracy has been established for 60 years at the time of this election. Meanwhile, Turkey is the least free of all of the countries that are surveyed in this Module. This limit in freedom of speech at the ballot box may also lay to limit freedom of association and expression for the people in the country, limiting the influence of one's vicinity and social group influence on one's political attitudes.

\begin{singlespace}
	\begin{table}[H]
		\centering
		%\def\arraystretch{1.5}
		\caption{\tb{Self-Placement Ideology - Middle East}}
		\begin{tabulary}{\linewidth}{l c c}
			\\
			\hline
			\tb{Place of Residence}&\tb{Israel} & \tb{Turkey} \\
			\hline
			Small Town&1.2822*** & .0228 \\
			&(.2427)  & (.3120)\\
			Suburban&1.5019*** &.5457 \\
			& (.2767)  & (.3326 )\\
			Urban &- & .2337\\
			&- & (.2764)\\
			Constant& 4.9193*** &5.7434*** \\
			& (,.2067)  & (.2268)\\
			N&912 & 965\\
			$R^2$&0.0371 & 0.0037 \\
			\hline 
		\end{tabulary}
		\\
		\e{Notes:} *p$<$.1, **p$<$.05. ***p$<$.01 \\
		\e{Reference:} A rural place of residence serves as the baseline for comparison
		\label{table10}
	\end{table}
\end{singlespace}

In the wake of the decolonization movement, countries in Africa are rather young in terms of the length of time that their democracy has been in place. In both of the cases displayed in Table \ref{table11}, it does not seem that the age of the democracy is the reason that place is likely to influence political ideology like other regions in the world. Kenya, a younger regime than South Africa, has a greater difference between urban and rural residents such that urban residents are more likely to be left-leaning at a larger factor (1.269 points on average to the left) than in other places in the world. 

\begin{singlespace}
	\begin{table}[H]
		\centering
		%\def\arraystretch{1.5}
		\caption{\tb{Self-Placement Ideology - Africa}}
		\begin{tabulary}{\linewidth}{l c c}
			\\
			\hline
			\tb{Place of Residence} & \tb{Kenya}& \tb{South Africa} \\
			\hline
			Urban&-1.2690*** &-.06719 \\
			&(.3701)  &(.1664)\\
			Constant&6.7690*** &6.4965*** \\
			& (.1932) & (.1392) \\
			N&506 & 967\\
			$R^2$&0.0228 &0.0002 \\
			\hline
		\end{tabulary}
		\\
		\e{Notes:} *p$<$.1, **p$<$.05. ***p$<$.01 \\
		\e{Reference:} A rural place of residence serves as the baseline for comparison
		\label{table11}
	\end{table}
\end{singlespace}

%Small Town&- &-\\
% &- & -\\
%Suburban&- & -\\
% & - &- \\

In this sample of countries from Asia, the data in Table \ref{table12}, suggests that neither Japan nor South Korea exhibit a significant trends towards having place influence a person's placement on the left-right scale. This is also visible in countries in the Pacific Island, and Australia. But in the cases of the countries displayed in table \ref{table13}, the more democratic a place is does not necessarily suggest that there is a greater division between urban and rural residents in terms of their political ideologies. The Philippines, for instance, is less free than Australia and New Zealand, but demonstrate a greater place divide in terms of political ideology.

\begin{singlespace}
	\begin{table}[H]
		\centering 
		%\def\arraystretch{1.5}
		\caption{\tb{Self-Placement Ideology - Asia}}
		\begin{tabulary}{\linewidth}{l c c}
			\\
			\hline
			\tb{Place of Residence} & \tb{Japan} & \tb{South Korea}\\
			\hline
			Small Town &.1898 & -.0347 \\
			&(.1463) & (.2738 )\\
			Suburban&.1717 & -.1176 \\
			& (.1574) & (.4343 )\\
			Urban & .2429* &.3140 \\
			& (.1365) & (.2518)\\
			Constant& 5.4179*** & 5.264*** \\
			& (.1210) & (.2158)\\
			N&1,563 & 717\\
			$R^2$& 0.0020 & 0.0052\\
			\hline
		\end{tabulary}
		\\
		\e{Notes:} *p$<$.1, **p$<$.05. ***p$<$.01 \\
		\e{Reference:} A rural place of residence serves as the baseline for comparison
		\label{table12}
	\end{table}
\end{singlespace}

\begin{landscape}
	\begin{table}
		\centering
		\def\arraystretch{1.5}
		\caption{\tb{Self-Placement Ideology - Pacific Islands}}
		\begin{tabulary}{\linewidth}{l c c c c}
			\\
			\hline
			\tb{Place of Residence}&\tb{Australia}&\tb{New Zealand (2011)}&\tb{New Zealand (2014)} &\tb{Philippines}\\
			\hline
			Small Town  & - .2108  & -.2283  &-.1436 &  .6419***  \\      
			& (.1540)   & (.2213) & (.2864)  & (.2060)    \\
			Suburban  & -.1433   & -   & -    & - .1095   \\ 
			& (.1570)   & -  & -    & (.3871)       \\
			Urban  & -.3005** & -.2653  & -.2151 & .3048** \\
			& (.1335) &(.2114)   & (.2406) & (.1526)      \\
			Constant   & 5.372***   & 5.754*** & 5.9674***  & 7.009***   \\
			&(.1219)&(.1826)&(.2207)&(.1030) \\
			N 	 & 3,363   &1,039    & 957   & 1,179  \\
			$R^2$  & 0.0018   &0.0016  &0.0035  &0.0097     \\
			\hline                                       
		\end{tabulary} 
		\\
		\e{Notes:} *p$<$.1, **p$<$.05. ***p$<$.01 \\
		\e{Reference:} A rural place of residence serves as the baseline for comparison
		\label{table13}
	\end{table}
\end{landscape}

\subsubsection{Summary: Place Matters?}

In our analysis for the influence of place of residence on the political attitudes of individuals in certain countries of the world, we generate rather mixed conclusions based on the data. When we consider the differences between neighboring countries in terms of differences in regime age and level of democracy, these factors, nonetheless produce mixed results as an answer to the question about whether place matters in influencing political ideology. From the overall trends, we have evidence to suggest that place of residence does matter, but it depends on contextual factors that extend beyond the information that a country's regime stability, as determined by age, and the freedoms people enjoy, as determined by level of democracy, can tell us about the underpinnings of these observations.

\section{General Trends For Both Objective Issue Stances and Self-Placement Ideology}

\begin{singlespace}
	\begin{table}[H]
		\centering
		\def\arraystretch{1}
		\caption{\tb{General Trends of Ideology}}
		\begin{tabulary}{\linewidth}{l c c}
			\\
			\hline
			\tb{Place of Residence}&\tb{Self-Placement} & \tb{Liberalism} \\
			\hline
			Small Town&-.1338***& -.1005*** \\    
			& (.0336) & (.0136)  \\
			Suburban & -.2591*** &-.1737 ***\\ 
			& (.0358) &(.0162) \\
			Urban   & -.0482 &-.1309***  \\
			& (.0300)  &(.123)  \\
			Constant   & 5.597*** &5.7164*** \\
			&(.0234) & (.0093)\\
			N  & 47,821 & 42,500 \\
			$R^2$	& 0.0011 & 0.0037 \\
			\hline                                       
		\end{tabulary}
		\\
		\e{Notes:} *p$<$.1, **p$<$.05. ***p$<$.01 \\
		\e{Reference:} A rural place of residence serves as the baseline for comparison
		\label{table102}
	\end{table}
\end{singlespace}

\begin{table}[h!]
	\centering
	%\def\arraystretch{1.5}
	\caption{\tb{All Macro Variables - General Trends}}
	\begin{tabulary}{\linewidth}{l c c}
		\\
		\hline
		\tb{Variable}&\tb{Self-Placement}&\tb{Liberalism} \\
		\hline
		\e{Place of Residence} & & \\
		Small Town &-.1000***&-.0018 \\
		&(.0339)&(.0130) \\
		Suburban& -.1517***& .0147 \\
		&(.0401)& (.0157) \\
		Urban&-.0517* & -.0420*** \\
		&(.0304) & (.0117) \\
		\e{Democracy}& -.1995*** & -.0421***\\
		&(.0110) & (.0040)\\
		\e{Regime Age} & -.0014*** &  -.0075***\\
		&(.0002) & (.0001)\\
		\e{Electoral Formula}&.0778*** & -.0440***\\
		&(.0189) & (.0070) \\
		\hline
		Constant& 7.3627*** & 6.5149*** \\
		&(.1067) & (.0391)\\
		N&46,555 & 41,373 \\
		$R^2$&0.0135& 0.1273 \\
		\hline
	\end{tabulary}
	\\
	\e{Notes:} *p$<$.1, **p$<$.05. ***p$<$.01 \\
	\e{Reference:} A rural place of residence serves as the baseline for comparison
	\label{table15}
\end{table}

\begin{table}[h!]
	\centering
	\caption{\tb{Ideology In Each Electoral Formula}}
	\begin{tabulary}{\linewidth}{l c c c}
		\\
		\hline
		\tb{Place of Residence}&\tb{Majoritarian}&\tb{Proportional} &\tb{Mixed} \\
		\hline
		Small Town&-.6640*** &.0608 &-.1600** \\
		&(.0766)&(.0435) &(.0727) \\
		Suburban&-.6389*** &-.1894***&-.1260  \\
		&(.1042) &(.0451) &(.1180) \\
		Urban&-.5781*** &-.0327 &.2655*** \\
		&(.0710) &(.0388) &(.0644) \\
		Constant& 5.8256*** &5.5408***&5.5562*** \\
		&(.0543) &(.0297) &(.0531) \\
		Countries&5&21&6 \\
		N&8,350&28,869 &10,602 \\
		$R^2$&0.0114&0.0011 &0.0052 \\
		\hline 
	\end{tabulary} 
	\\ 
	\e{Notes:} *p$<$.1, **p$<$.05. ***p$<$.01 \\
	\e{Reference:} A rural place of residence serves as the baseline for comparison
	\label{table16}
\end{table}

\end{document}
