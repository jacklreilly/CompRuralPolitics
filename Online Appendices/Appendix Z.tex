\documentclass[12pt, titlepage]{article}

\usepackage{fullpage} %full page typesetting
\usepackage{setspace} %allows for non-singlespacing
\usepackage{graphicx} %graphics capabilities
\usepackage{latexsym} %extra symbols
\usepackage{rotating} %rotation for figures
\usepackage{longtable} %tables that fill more than a single page
%\usepackage{hyperref} %hypertext links in the document
\usepackage{natbib} %better bibliographies
\usepackage{authblk} %author and affiliation in opening
\usepackage{mathpazo} %use palatino font, rather than times
\usepackage{appendix}
\usepackage{lscape}
\usepackage{tabulary}
\usepackage[nottoc]{tocbibind}
\usepackage[colorlinks=true,linkcolor=blue,citecolor=cyan]{hyperref}
\usepackage{ifthen}
\usepackage{float}


\title{\tb{Place of Residence and Political Attitudes in Democracies Worldwide \\ {\large Online Appendix Z-- Types of Elections} }}

\author{Jennifer Lin}
\affil{New College of Florida}

\newcommand\e{\emph}
\newcommand\tb{\textbf}
\newcommand\un{\underline}
\newcommand\txt{\texttt}

\doublespacing 

\begin{document}
	
	\begin{singlespace}
		\maketitle
	\end{singlespace}

% Table for D1015 - election type
\begin{table}
	\centering
	\resizebox{\columnwidth}{!}
	\def\arraystretch{1.5}
	\caption{\tb{Types of Election by Polity}}
	\begin{tabulary}{\textwidth}{l c c c c} 
		\hline
		\tb{Country}&\tb{Year}&\tb{Legislative}&\tb{Both}&\tb{Presidential}\\
		\hline
		Argentina&2015&-&X&-\\
		Australia&2013&X&-&-\\
		Austria&2013&X&-&-\\
		Bulgaria&2014&X&-&-\\
		Czech Republic&2013&X&-&-\\
		Finland&2015&X&-&-\\
		France&2012&-&-&X\\
		Germany&2013&X&-&-\\ 
		Great Britain&2015&X&-&-\\
		Greece&2012/2015&X&-&-\\
		Iceland&2013&X&-&-\\
		Ireland&2011&X&-&-\\
		Israel&2013&X&-&-\\
		Japan&2013&X&-&-\\
		Kenya&2013&-&X&-\\
		Latvia&2011/2014&X&-&-\\
    	Mexico&2012&-&X&-\\
		Mexico&2015&X&-&-\\
		Montenegro&2012&X&-&-\\
		Norway&2013&X&-&-\\
		New Zealand&2011/2014&X&-&-\\
		Peru&2016&-&X&-\\
		Philippines&2016&-&X&-\\
		Poland&2011&X&-&-\\
		Portugal&2015&X&-&-\\	
		Romania&2012&X&-&-\\
		Romania&2014&-&-&X\\
		Serbia&2012&-&X&-\\
		Slovakia&2016&X&-&-\\
		Slovenia&2011&X&-&-\\
		South Africa&2014&X&-&-\\ 
		South Korea&2012&X&-&-\\
		Sweden&2014&X&-&-\\
		Switzerland&2011&X&-&-\\
		Turkey&2015&X&-&-\\
		\hline
	\end{tabulary} \\
	\e{Source:} Comparative Study of Electoral Systems Module IV 
	\label{table1}
\end{table}

%Brazil&2014&-&X&-\\
%Canada&2011&X&-&-\\
%Canada&2015&X&-&-\\
%Hong Kong&2012&X&-&-\\
%United States&2012&-&X&-\\
%Taiwan&2012&-&X&-\\
%Thailand&2011&X&-&-\\

Table \ref{table1} shows the types of elections that are held within each of the polities that are represented in the Comparative Study of Electoral Systems Module IV data. From this table, we see that many of the countries are holding legislative, or parliamentary elections to some extent during the period between 2011 and 2016. This table provides some background for the context in each of the polities that will be included in the analysis.

\end{document}